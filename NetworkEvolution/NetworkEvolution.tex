% This is samplepaper.tex, a sample chapter demonstrating the
% LLNCS macro package for Springer Computer Science proceedings;
% Version 2.20 of 2017/10/04
%
\documentclass[runningheads]{llncs}
%
\usepackage{fixltx2e}
\usepackage[american]{babel}
\usepackage[utf8]{inputenc}
\usepackage{csquotes}
\usepackage{graphicx}
\usepackage{xcolor}
\usepackage{hyperref}
\usepackage{url}
\usepackage[T1]{fontenc}
\usepackage{lmodern}
\usepackage{microtype}
\usepackage{eurosym}
\usepackage{biblatex}
\addbibresource{bibliography.bib}
\addbibresource{rpackages.bib}
% Used for displaying a sample figure. If possible, figure files should
% be included in EPS format.
%
% If you use the hyperref package, please uncomment the following line
% to display URLs in blue roman font according to Springer's eBook style:
%\renewcommand\UrlFont{\color{blue}\rmfamily}
\hypersetup{breaklinks=true,
            bookmarks=true,
            pdfauthor={},
            pdftitle={Influencing Factors on Social Network Evolution},
            colorlinks=true,
            citecolor=blue,
            urlcolor=blue,
            linkcolor=magenta,
            pdfborder={0 0 0}}
\urlstyle{same}


\usepackage{longtable,booktabs}
\newcommand{\eg}{e.\,g.,\ }
\newcommand{\ie}{i.\,e.,\ }
%\setkeys{Gin}{width=\maxwidth,height=\maxheight,keepaspectratio}

\IfFileExists{upquote.sty}{\usepackage{upquote}}{}
% use microtype if available
\IfFileExists{microtype.sty}{%
\usepackage{microtype}
\UseMicrotypeSet[protrusion]{basicmath} % disable protrusion for tt fonts
}{}

\providecommand{\tightlist}{%
  \setlength{\itemsep}{0pt}\setlength{\parskip}{0pt}}

\begin{document}
%
\title{Influencing Factors on Social Network Evolution}
%
%\titlerunning{Abbreviated paper title}
% If the paper title is too long for the running head, you can set
% an abbreviated paper title here
%
\input{anonauthors.tex}

%
\maketitle              % typeset the header of the contribution
%
\begin{abstract}
The abstract should briefly summarize the contents of the paper in
150--250 words.

	\keywords{First keyword \and Second keyword \and Another keyword.}
\end{abstract}
%
%
%
\hypertarget{introduction}{%
\section{Introduction}\label{introduction}}

In the past decade the evolution of the internet and social media
platforms raised new forms of social networks that changed our
interpersonal communication and the methods of information procurement
considerably. It has become very easy to connect to existing friends
online, look for new friends and exchange information with them for
example using platforms like Facebook or Twitter. As the formation of an
individual's opinion is based on all available resources it is important
to understand how the information received in online social networks is
embedded into the process of opinion formation and how people behave in
such networks.

\hypertarget{theory}{%
\section{Theory}\label{theory}}

Existing research states that the vast availability of similar-minded
people in online social networks leads us to enclose ourselves in
so-called echo chambers and to disconnect from people who are too
different from us. This leads us to reinforce our opinion solely through
finding others that think similar. If this reinforcement would continue,
people would be separated into different camps quickly and would not be
able to agree on each other anymore. As most people prefer to make
compromises it is interesting to have a closer look on the thresholds
which let them keep in touch with people who they don't agree on and how
a variation of these thresholds changes the overall picture. Therefore,
we created an agent-based model that allows us to simulate the desired
behaviors and to compare the resulting network structures.

To reproduce an online social network in a simulation we must rely on a
network generator that is comparable to a real network. As social media
platforms such as Facebook and Twitter are close to a scale-free network
in their overall network structure and follow a powerlaw distribution,
the Barabasi-Albert network generator is most suitable for generating a
realistic network topology. Within a Barabasi-Albert network there
exists a little amount of very well-connected hubs while most nodes have
only few connections to others. The Barabasi-Albert generator provides
several parameters like the initial network size, the number of new
nodes that are added to an initial network and the number of edges which
are created by the joining nodes to existing nodes using preferential
attachment. This allows us to get the simulation close to a real social
network structure.

\hypertarget{method}{%
\section{Method}\label{method}}

We chose the programming language Julia to conduct our research. With
the LightGraphs package, this language provides performant network
simulation and the required network generators for our agent-based
model. It is also possible to implement batch runs that are based on the
same random seed so that the network evolution following different
parameters can be analyzed subsequently.

In our research, we focused on the variation of limited parameters for
answering our research questions:

\begin{itemize}
\tightlist
\item
  Size of the network: How do network and opinion dynamics interplay
  with the size of a social network?
\item
  Adding friends: What is the difference between randomly making friends
  in the network and choosing only from the friends of existing friends?
\item
  Removing friends: How does the threshold for accepting opinion
  differences interfere with the overall opinion and network dynamics?
  The distribution of opinions throughout the agents was not varied, but
  uniformly distributed, because their variation would have blurred the
  effect of the examined parameters on the network evolution.
\end{itemize}

To analyze the effect of our parameters, we chose different approaches
of social network analysis and evaluated the resulting networks and
their nodes regarding their degree, centrality, communality, diameter,
and clustering coefficient.

\hypertarget{results}{%
\section{Results}\label{results}}

Our simulation study comprised in total 18 simulation runs that show the
impact and relationships of the varied factors on the network evolution
and opinion dynamics.

\hypertarget{first-section}{%
\section{First Section}\label{first-section}}

\hypertarget{a-subsection-sample}{%
\subsection{A Subsection Sample}\label{a-subsection-sample}}

Please note that the first paragraph of a section or subsection is not
indented. The first paragraph that follows a table, figure, equation
etc. does not need an indent, either.

Subsequent paragraphs, however, are indented.

\hypertarget{sample-heading-third-level}{%
\subsubsection{Sample Heading (Third
Level)}\label{sample-heading-third-level}}

Only two levels of headings should be numbered. Lower level headings
remain unnumbered; they are formatted as run-in headings.

\hypertarget{sample-heading-fourth-level}{%
\paragraph{Sample Heading (Fourth
Level)}\label{sample-heading-fourth-level}}

The contribution should contain no more than four levels of headings.
Table~\ref{tab1} gives a summary of all heading levels.

Another nice feature are shortcuts for \eg and \ie 

\hypertarget{references}{%
\section{References}\label{references}}

You can cite any paper in parenthesis as following
\autocite{valdez2017priming} or inline saying that
\textcite{valdez2017priming} found something. Multiple citations are
possible as well~\autocite{valdez2017priming,valdez2019users}.

You can refer to other sections by kebab-casing to
section~\ref{a-subsection-sample}. You can easily cite an r-package
directly in the text by using the \texttt{cite\_pkg} function from the
package \texttt{rmdtemplates}~\autocite{R-rmdtemplates}.

\hypertarget{environments}{%
\section{Environments}\label{environments}}

The environments \enquote{definition}, \enquote{lemma},
\enquote{proposition}, \enquote{corollary}, \enquote{remark}, and
\enquote{example} are defined in the LLNCS document class as well.

\hypertarget{theorems}{%
\subsection{Theorems}\label{theorems}}

\begin{theorem}
This is a sample theorem. The run-in heading is set in bold, while
the following text appears in italics. Definitions, lemmas,
propositions, and corollaries are styled the same way.
\end{theorem}

\hypertarget{equations}{%
\subsection{Equations}\label{equations}}

\begin{equation}
x + y = z
\end{equation}

\hypertarget{tables}{%
\subsection{Tables}\label{tables}}

You can get the non breaking space in RStudio by pressing ALT+SPACE. You
can refer to tables by using Table~\ref{tab:table_1}.

\begin{longtable}[]{@{}rrrrl@{}}
\caption{Test\label{tab:table_1}}\tabularnewline
\toprule
Sepal.Length & Sepal.Width & Petal.Length & Petal.Width &
Species\tabularnewline
\midrule
\endfirsthead
\toprule
Sepal.Length & Sepal.Width & Petal.Length & Petal.Width &
Species\tabularnewline
\midrule
\endhead
5.1 & 3.5 & 1.4 & 0.2 & setosa\tabularnewline
4.9 & 3.0 & 1.4 & 0.2 & setosa\tabularnewline
4.7 & 3.2 & 1.3 & 0.2 & setosa\tabularnewline
4.6 & 3.1 & 1.5 & 0.2 & setosa\tabularnewline
5.0 & 3.6 & 1.4 & 0.2 & setosa\tabularnewline
5.4 & 3.9 & 1.7 & 0.4 & setosa\tabularnewline
\bottomrule
\end{longtable}

\hypertarget{inline-latex-tables}{%
\subsubsection{Inline Latex Tables}\label{inline-latex-tables}}

You can directly add latex tables.

\begin{table}
\caption{Table captions should be placed above the
tables.}\label{tab1}
\begin{tabular}{|l|l|l|}
\hline
Heading level &  Example & Font size and style\\
\hline
Title (centered) &  {\Large\bfseries Lecture Notes} & 14 point, bold\\
1st-level heading &  {\large\bfseries 1 Introduction} & 12 point, bold\\
2nd-level heading & {\bfseries 2.1 Printing Area} & 10 point, bold\\
3rd-level heading & {\bfseries Run-in Heading in Bold.} Text follows & 10 point, bold\\
4th-level heading & {\itshape Lowest Level Heading.} Text follows & 10 point, italic\\
\hline
\end{tabular}
\end{table}

\hypertarget{figures}{%
\subsection{Figures}\label{figures}}

You can refer to tables by using Figure~\ref{fig:fig1}.

\begin{figure}
\includegraphics[width=1\linewidth]{NetworkEvolution_files/figure-latex/fig1-1} \caption{This is the text caption under the figure}\label{fig:fig1}
\end{figure}

\hypertarget{acknowledgements}{%
\section*{Acknowledgements}\label{acknowledgements}}
\addcontentsline{toc}{section}{Acknowledgements}

We would like to thank xyz. We would further like to thank the authors
of the packages we have used. We used the following packages to create
this document: \texttt{knitr}~\autocite{R-knitr},
\texttt{tidyverse}~\autocite{R-tidyverse},
\texttt{rmdformats}~\autocite{R-rmdformats},
\texttt{kableExtra}~\autocite{R-kableExtra},
\texttt{scales}~\autocite{R-scales}, \texttt{psych}~\autocite{R-psych},
\texttt{rmdtemplates}~\autocite{R-rmdtemplates}.


%
% ---- Bibliography ----
%
% BibTeX users should specify bibliography style 'splncs04'.
% References will then be sorted and formatted in the correct style.
%
%\bibliographystyle{splncs04}
%\bibliography{bibliography,rpackages}
\printbibliography



\end{document}
